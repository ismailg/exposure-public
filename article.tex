% Options for packages loaded elsewhere
\PassOptionsToPackage{unicode}{hyperref}
\PassOptionsToPackage{hyphens}{url}
\documentclass[
]{article}
\usepackage{xcolor}
\usepackage[margin=1in]{geometry}
\usepackage{amsmath,amssymb}
\setcounter{secnumdepth}{5}
\usepackage{iftex}
\ifPDFTeX
  \usepackage[T1]{fontenc}
  \usepackage[utf8]{inputenc}
  \usepackage{textcomp} % provide euro and other symbols
\else % if luatex or xetex
  \usepackage{unicode-math} % this also loads fontspec
  \defaultfontfeatures{Scale=MatchLowercase}
  \defaultfontfeatures[\rmfamily]{Ligatures=TeX,Scale=1}
\fi
\usepackage{lmodern}
\ifPDFTeX\else
  % xetex/luatex font selection
\fi
% Use upquote if available, for straight quotes in verbatim environments
\IfFileExists{upquote.sty}{\usepackage{upquote}}{}
\IfFileExists{microtype.sty}{% use microtype if available
  \usepackage[]{microtype}
  \UseMicrotypeSet[protrusion]{basicmath} % disable protrusion for tt fonts
}{}
\makeatletter
\@ifundefined{KOMAClassName}{% if non-KOMA class
  \IfFileExists{parskip.sty}{%
    \usepackage{parskip}
  }{% else
    \setlength{\parindent}{0pt}
    \setlength{\parskip}{6pt plus 2pt minus 1pt}}
}{% if KOMA class
  \KOMAoptions{parskip=half}}
\makeatother
\usepackage{longtable,booktabs,array}
\usepackage{calc} % for calculating minipage widths
% Correct order of tables after \paragraph or \subparagraph
\usepackage{etoolbox}
\makeatletter
\patchcmd\longtable{\par}{\if@noskipsec\mbox{}\fi\par}{}{}
\makeatother
% Allow footnotes in longtable head/foot
\IfFileExists{footnotehyper.sty}{\usepackage{footnotehyper}}{\usepackage{footnote}}
\makesavenoteenv{longtable}
\usepackage{graphicx}
\makeatletter
\newsavebox\pandoc@box
\newcommand*\pandocbounded[1]{% scales image to fit in text height/width
  \sbox\pandoc@box{#1}%
  \Gscale@div\@tempa{\textheight}{\dimexpr\ht\pandoc@box+\dp\pandoc@box\relax}%
  \Gscale@div\@tempb{\linewidth}{\wd\pandoc@box}%
  \ifdim\@tempb\p@<\@tempa\p@\let\@tempa\@tempb\fi% select the smaller of both
  \ifdim\@tempa\p@<\p@\scalebox{\@tempa}{\usebox\pandoc@box}%
  \else\usebox{\pandoc@box}%
  \fi%
}
% Set default figure placement to htbp
\def\fps@figure{htbp}
\makeatother
% definitions for citeproc citations
\NewDocumentCommand\citeproctext{}{}
\NewDocumentCommand\citeproc{mm}{%
  \begingroup\def\citeproctext{#2}\cite{#1}\endgroup}
\makeatletter
 % allow citations to break across lines
 \let\@cite@ofmt\@firstofone
 % avoid brackets around text for \cite:
 \def\@biblabel#1{}
 \def\@cite#1#2{{#1\if@tempswa , #2\fi}}
\makeatother
\newlength{\cslhangindent}
\setlength{\cslhangindent}{1.5em}
\newlength{\csllabelwidth}
\setlength{\csllabelwidth}{3em}
\newenvironment{CSLReferences}[2] % #1 hanging-indent, #2 entry-spacing
 {\begin{list}{}{%
  \setlength{\itemindent}{0pt}
  \setlength{\leftmargin}{0pt}
  \setlength{\parsep}{0pt}
  % turn on hanging indent if param 1 is 1
  \ifodd #1
   \setlength{\leftmargin}{\cslhangindent}
   \setlength{\itemindent}{-1\cslhangindent}
  \fi
  % set entry spacing
  \setlength{\itemsep}{#2\baselineskip}}}
 {\end{list}}
\usepackage{calc}
\newcommand{\CSLBlock}[1]{\hfill\break\parbox[t]{\linewidth}{\strut\ignorespaces#1\strut}}
\newcommand{\CSLLeftMargin}[1]{\parbox[t]{\csllabelwidth}{\strut#1\strut}}
\newcommand{\CSLRightInline}[1]{\parbox[t]{\linewidth - \csllabelwidth}{\strut#1\strut}}
\newcommand{\CSLIndent}[1]{\hspace{\cslhangindent}#1}
\setlength{\emergencystretch}{3em} % prevent overfull lines
\providecommand{\tightlist}{%
  \setlength{\itemsep}{0pt}\setlength{\parskip}{0pt}}
\usepackage{lscape}
\usepackage{booktabs}
\usepackage{longtable}
\usepackage{array}
\usepackage{multirow}
\usepackage{wrapfig}
\usepackage{float}
\usepackage{colortbl}
\usepackage{pdflscape}
\usepackage{tabu}
\usepackage{threeparttable}
\usepackage{threeparttablex}
\usepackage[normalem]{ulem}
\usepackage{makecell}
\usepackage{xcolor}
\usepackage{bookmark}
\IfFileExists{xurl.sty}{\usepackage{xurl}}{} % add URL line breaks if available
\urlstyle{same}
\hypersetup{
  pdftitle={When Forgiveness Backfires: Rejection Sensitivity and Cooperative Behavior Following Exposure to Adaptive Forgiving Agents},
  pdfauthor={Ismail Guennouni1,2,3,4*, Georgia Koppe1,2,3\^{}\textbackslash dagger, Christoph Korn4\^{}\textbackslash dagger},
  hidelinks,
  pdfcreator={LaTeX via pandoc}}

\title{When Forgiveness Backfires: Rejection Sensitivity and Cooperative Behavior Following Exposure to Adaptive Forgiving Agents}
\author{Ismail Guennouni\textsuperscript{1,2,3,4}*, Georgia Koppe\textsuperscript{1,2,3}\(^\dagger\), Christoph Korn\textsuperscript{4}\(^\dagger\)}
\date{}

\begin{document}
\maketitle

\small

\textsuperscript{1} \emph{Department of Psychiatry and Psychotherapy, Central Institute of
Mental Health, Medical Faculty Mannheim, Heidelberg University,
Mannheim, Germany}

\textsuperscript{2} \emph{Interdisciplinary Center for Scientific Computing, Faculty of
Mathematics and Computer Science, Heidelberg University, Heidelberg,
Germany}

\textsuperscript{3} \emph{Hector Institute for AI in Psychiatry, Central Institute of Mental
Health, Medical Faculty Mannheim, Heidelberg University, Mannheim
Germany}

\textsuperscript{4} \emph{Department of General Psychiatry, Section Social Neuroscience,
Heidelberg University, Germany}

\(^\dagger\) \emph{Joint last author}

\textsuperscript{*} \emph{Corresponding author. Address: Central institute of Mental Health,
J5, Mannheim, Germany. Email:
\href{mailto:ismail.guennouni@zi-mannheim.de}{\nolinkurl{ismail.guennouni@zi-mannheim.de}}}

\pagebreak

\textbf{Abstract:}

Can exposure to forgiving partners improve interpersonal cooperation? Attachment theory suggests positive relational experiences can correct negative internal working models, but individuals high in Rejection Sensitivity (RS)---characterized by anxious expectations of rejection---may struggle to benefit from such experiences. We tested this using a randomized experiment (N = 206) in which participants played repeated trust games with HMM-based artificial agents that simulate human-like trust dynamics. After a baseline game, participants were exposed to either forgiving agents (Manipulation) or standard agents (Control), then played a final game with a standard agent. Contrary to predictions, forgiveness exposure \emph{reduced} subsequent cooperation---participants appeared to perceive the standard post-exposure agent as less cooperative by comparison (a negative contrast effect). Critically, RS moderated specific behavioral patterns but not overall cooperation levels: high RS participants failed to recover cooperation after trust violations and became \emph{less} responsive to partner behavior following exposure, whereas low RS participants showed normal recovery and became \emph{more} responsive. These findings suggest that positive relational experiences do not universally promote cooperation, and that high RS individuals may require interventions targeting their capacity to update expectations rather than simply providing positive experiences.

\vspace{1cm}

\textbf{Keywords} : Interpersonal functioning; Rejection Sensitivity;
Forgiveness Intervention; Trust-based Cooperation; Hidden Markov Models

\vspace{1cm}

\textbf{General Scientific Summary:}

This study found that exposing people to forgiving partners in economic games unexpectedly decreased their subsequent cooperation---likely because normal partners seemed less cooperative by comparison. Individuals high in rejection sensitivity showed a distinct pattern: while they detected trust violations as readily as others, they failed to restore cooperation afterward and became less responsive to their partner's behavior. In contrast, those low in rejection sensitivity appeared to learn from the positive exposure and became more reciprocal. These findings suggest that simply providing positive social experiences may not benefit everyone equally, and that individuals prone to rejection sensitivity may need targeted support to translate positive experiences into lasting behavioral change.

\pagebreak

\section{Introduction}\label{introduction}

Trust is fundamental to human social interactions, facilitating seamless
relations at both interpersonal and intergroup levels. The study of
psychopathology has linked deficits in trust-based constructs to the
development of mental health disorders (Fonagy \& Campbell, 2017).
Individuals with personality disorders (PD) often struggle to form and
maintain social connections, a difficulty reflected in uncooperative
behaviors -- a marker for the severity of PD symptoms
(Herpertz \& Bertsch, 2014; Mulder et al., 1999).

One explanation for such social challenges lies in early caregiver
experiences. Attachment theory (Bowlby, 1978) suggests that
the quality of these relationships shapes our capacity for secure
attachments and trust. Individuals with higher levels of insecure
attachment may recall negative trust-related experiences more easily,
report fewer positive trust experiences, and use less constructive
coping strategies when trust is broken (Mikulincer, 1998).
These insecure attachment patterns are often associated with heightened
rejection sensitivity (RS), a tendency to anxiously expect, readily
perceive, and intensely react to rejection (Downey et al., 1997; Downey \& Feldman, 1996). RS has been linked to the development of various
mental health conditions, including depression, anxiety, personality
disorders, and self-harm (Gao et al., 2017). Individuals high in
RS show attentional biases towards social threat cues, which may
contribute to difficulties in social interactions
(Berenson et al., 2009). A recent meta-analysis revealed prosocial
behavior and interpersonal trust as two key processes of interpersonal
functioning that are markedly impaired in PDs and which are likely to
contribute to interpersonal dysfunction in this population
(Hepp \& Niedtfeld, 2022). The interaction of RS and trust-based
constructs has been explored, particularly in Borderline Personality
Disorder (BPD). Miano et al. (2013) and Richetin et al. (2018) found
that RS mediated the relationship between BPD features and lower trust
appraisal. Abramov et al. (2022) found that higher baseline feelings
of rejection in individuals with BPD predict slower trust formation and
less pronounced declines in trust following trust violations during the
trust game. However, the interaction between \emph{reciprocity} and RS hasn't
been studied as extensively, leaving a gap in our understanding of how
these constructs might interplay.

Given that RS may be a manifestation of maladaptive attachment styles,
it is important to explore whether exposure to consistently forgiving
interaction partners could reshape interpersonal expectations and
behaviors. The \emph{corrective experience hypothesis}, rooted in attachment
theory, suggests that new positive relational experiences can modify
internal working models of relationships (Bowlby, 1988). Research
on social learning (Bandura, 1977) similarly demonstrates that
individuals model the behavior of those around them, and exposure to
cooperative peers promotes cooperative behavior
(Fowler \& Christakis, 2010). In the repeated trust game (RTG) paradigm,
cycles of reciprocated trust enhance cooperative behaviors even among
initially distrustful individuals (King-Casas et al., 2005). This
perspective predicts that exposure to forgiving partners should increase
subsequent cooperation, as participants internalize more positive
expectations about social interactions. Given that high RS individuals
show heightened sensitivity to social cues, they might be particularly
responsive to such corrective experiences.

In this study, we use a randomized controlled online experiment to test
whether exposing participants with varying RS levels to forgiving and
more cooperative co-players results in more trustworthy behavior and a
repair of potential breakdowns in RTG cooperation. To simulate realistic
social interaction while maintaining a high degree of experimental
control, we take a novel paradigmatic approach: We use generative models
of how humans play the RTG to design an agent that plays the role of the
investor, based on Hidden Markov Models (HMMs) fitted to real players'
data. A key aspect of these agents is that their actions depend on a
latent ``trust state'' which reacts dynamically to the trustees' returns,
simulating real-life trust-building scenarios. An advantage of having
such a generative model of behavior is the possibility of controlling
different aspects of the agent's strategy such as its general policy,
the propensity to cooperate actively, or the propensity to trust again
after breakdowns of cooperation. To further mimic real-world
interactions and examine participants' responses to one-off breakdowns
of cooperation, we incorporate occasional pre-programmed low investments
by the agent.

We pre-screened participants for high or low RS using a validated
questionnaire, then assigned them exclusively to the trustee role in a
series of trust games. After playing a 15-round RTG with a human-like
HMM investor, they were randomly assigned to either a Control or
Manipulation condition. In the Manipulation condition, participants were
exposed over three RTGs to HMM investors designed with a limited
propensity for retaliation. In the Control condition, participants
played three RTGs against the same human-like HMM. After this exposure
phase, all participants played another 15-round RTG with a human-like
HMM investor, similar to the one in the pre-exposure phase.

Based on the corrective experience hypothesis, we predicted that
forgiveness exposure would increase subsequent cooperation, with high RS
individuals potentially benefiting from positive relational experiences
that challenge their negative expectations. We examined both overall
effects of the manipulation and differential responses based on RS, with
particular attention to how participants respond to trust violations
before and after the exposure phase.

\section{Methods}\label{methods}

\subsection{Participants}\label{participants}

To have participants with large differences in RS, a total of 1195
participants were pre-screened on the Prolific Academic platform
(prolific.co) using the Rejection Sensitivity Questionnaire (RSQ) to
finally select two similarly sized groups: One with high RS (RSQ score
\textgreater{} 15, N=103) and the other with low RS (RSQ score \textless{} 10, N=103)
totalling 206 participants (56\% female). These were then invited through
prolific to take part in the main experiment. The required sample size
was determined using an \emph{a priori} power analysis to have an 80\%
probability to detect a small effect size (Cohen's f = 0.10) for a
within-between interaction with a 5\% type I error rate in a repeated
measures ANOVA. The sample size calculation assumed 2 groups, 2
measurement per group and was performed using the G*Power software
(Faul et al., 2009). The mean age of participants was 34.6 years,
with an 11.9 years standard deviation. The majority of participants
identified ethnically as White (\(80\)\%). The online cohort registered 30
unique countries of birth with the most frequent being the U.K (\(33\)\%)
followed by Poland (\(10\)\%) and Portugal (\(10\)\%). Participants were paid
a fixed fee of £6 plus a bonus payment dependent on their performance
that averaged £0.5. Data was collected over multiple sessions between
December 2023 and February 2024.

\subsection{Design and Procedure}\label{design-and-procedure}

The experiment had a 2 (Condition: Manipulation or Control) by 2 (RS :
High or Low) by 2 (Phase: Trust-Game Pre-Exposure, Trust-Game Post-Exposure) design, with repeated measures on the Phase factor (Figure
\ref{fig:HMMPanels}.A). Participants within each pre-screened RS group
were randomly assigned to one of the two levels of the Condition factor,
resulting in 101 participants in the Manipulation condition and 105 in
the Control condition. The games were designed and implemented online
using Empirica v1 (Almaatouq et al., 2021). The planned experiment
received approval from the University of Heidelberg's Medical Faculty
ethics commission (ID:S-708/2023) and the experiment was performed in
accordance with the ethics board guidelines and regulations. All
participants provided informed consent prior to their participation.

\subsection{Tasks and Measures}\label{tasks-and-measures}

\subsubsection{Repeated Trust Game and HMM Investor}\label{repeated-trust-game-and-hmm-investor}

Participants played a 15-round RTG (Joyce et al., 1995) in the trustee
role against a computer-programmed investor. On each round the investor
is endowed with 20 units and decides how much of that endowment to
invest. This investment is tripled and the trustee then decides how to
split this tripled amount between them and the investor. If the trustee
returns more than one third of the amount, the investor makes a gain.
Each player was represented with an icon with the participant always on
the left of the screen and the co-player on the right. The participants
were able to choose the icon that represents them at the start of the
experiment. The icon representing the co-player changed at the start of
each new game, to simulate a new interaction partner. Participants were
not told they were facing computerised co-players. We chose to simulate
the behavior of a human interaction partner through allowing for a delay
whilst pairing with new opponents as the start of each game as well as
programming the agents to respond during each round after a varying time
lapse (randomly chosen between 5 and 10 seconds).

The computerised investor consisted of a hidden Markov model (HMM)
trained on an independent existing behavioral RTG data set of human
investors. This data-driven approach thus sought to learn an investor
strategy that mimics human-like interactions. The data set used for
training consists of 388 ten round games with the same player (full
details can be found in the Supplementary Information). On this data
set, the HMM was inferred with three latent states that could be
interpreted as reflecting a ``low-trust'', a ``medium-trust'', and a
``high-trust'' state. A separate output distribution, that maps each HMM
state onto possible investments from 0 to 20 separately, is learned
(Figure \ref{fig:HMMPanels}.B). In analogy to the latent states, these
distributions can be interpreted as reflecting ``low-trust'',
``medium-trust'', or ``high-trust'' dispositions. Finally, the HMM is
specified by transition probabilities that describe the transition
between states. The probability of these transitions was modelled as a
function of their net return (i.e return - investment) in the previous
round (see Figure \ref{fig:HMMPanels}.C)). The initial state for the
HMM investor in each instance of the game was set to the ``mid-trust''
state. Details on how the HMM state conditional probabilities and
transition functions are specified can be found in the supplement.

In order to instigate a potential breakdown of trust, thereby allowing
us to probe efforts to repair it, the computerised agent was programmed
to provide a low investment on round 12 (pre-exposure) and round 13
(post-exposure). On all other rounds, the investor's actions were
determined by randomly drawing an investment from the state-conditional
distribution, with the state over rounds determined by randomly drawing
the next state from the state-transition distribution as determined from
the net return on the previous round (disregarding the net return
immediately after the pre-programmed low investment rounds).

\subsection{Manipulation}\label{manipulation}

In all phases of the RTG other than the `Exposure phase' (Figure
\ref{fig:HMMPanels}.A), participants interacted with this human-like
HMM. In the `Manipulation' Condition of the exposure phase, however, the
parameters of this HMM were adjusted to design a `forgiving' and
ultimately more cooperative agent. To achieve this, we changed the state
transition probabilities of the HMM such that it becomes impossible for
it to remain in a low trust state, effectively setting the transition
probability for remaining in a ``low-trust'' state to 0. The resulting
transition function is shown in Figure \ref{fig:HMMPanels}.D. The
policies conditional on the latent states and the transition function in
the other latent states remain unchanged.

\subsection{Procedure}\label{procedure}

At the start of the experiment, participants provided informed consent
and were instructed the study would consist of three phases in which
they would face a different other player. Participants were told their
goal was to maximise the number of points in all phases. They were not
told the number of rounds of each phase. Participants were randomly
assigned to either a Control or Manipulation condition. The timeline of
the experiment is shown in Figure \ref{fig:HMMPanels}.A. Phase one
(``pre'') consisted of a 15 round RTG in which participants took the role
of trustee, facing the same investor over all 15 rounds. On each round,
after being informed about the amount sent by the investor participants
decided how much of the tripled investment to return to the investor,
before continuing to the next round. Phase 2 (``exposure'') consisted of
three 7-round RTGs. Participants in the Manipulation condition faced the
forgiving HMM investor and rated the agent on the same attributes as in
the pre-exposure phase. Those in the Control condition faced the
same human-like HMM agent as in the ``pre'' phase and rated each co-player
on the same attributes. To keep the experience similar to the ``pre''
phase, the agent in the Control condition was also designed to send a
very low investment in round 5 of each of the three games. In the
post-exposure phase (``post''), participants in both conditions faced
the same human-like HMM as in ``pre'' phase.

At the beginning of each game in all three phases, participants were
told they would face a new player and had to wait to be paired with an
available co-player. This simulated the waiting time in real social
interaction tasks. After completing each RTG in each phase, participants
rated how cooperative and forgiving they perceived the co-player to be,
and whether they would like to play with them again (all on a scale from
1 to 10 with 10 being the most positive rating). After completing the
three game phases, participants then completed the Levels of Personality
Functioning Scale Brief-Form (LPFS-BF) questionnaire
(Weekers et al., 2019). This is a self-report measure designed to assess
core elements of personality functioning as defined in the Alternative
Model for Personality Disorders in the DSM-5
(American Psychiatric Association, 2013), and provides a
dimensional assessment of personality functioning, which complements the
categorical approach of RS. Finally, participants were asked whether
they thought the other players were human or computer agents, to probe
how well the agent can mimic human behavior, then debriefed and thanked
for their participation.

\begin{landscape}
\begin{figure}

{\centering \includegraphics[width=0.95\linewidth]{article_files/figure-latex/HMMPanels-1} 

}

\caption{A: Experiment timeline. Participants (trustees) played RTGs with HMM investor agents. The investor sends investments (multiplied by 3) and participants decide returns. Conditions differ in exposure phase agents. B-D: The artificial investor is a three-state HMM fitted to human data. B: Investment distributions by latent state. C: Transition probabilities to states at t+1 as a function of net return at t; each panel shows a different starting state. D: Forgiving HMM transitions from low-trust state---unlike the human-like HMM, it always exits low-trust and favors high-trust transitions.}\label{fig:HMMPanels}
\end{figure}
\end{landscape}

\subsection{Statistical Analysis}\label{statistical-analysis}

We analyzed participants' behavior in the RTG using linear mixed-effects models. First, to examine the effect of the manipulation, we modeled the percentage return (percentage of tripled investment returned to investor) as a function of Phase (RTG game pre vs.~post-exposure), Condition (Manipulation vs.~Control), Investment, and RS (High vs Low RS group), including all interactions as fixed effects. This model included player-wise random intercepts and slopes for Phase. Second, we analyzed behavior during the Exposure phase specifically, modeling returns with Condition, Investment, and RS and their interactions as fixed effects, along with player-wise random intercepts. Third, to verify the consistency of the HMM agent, we modeled the investments sent by the computerized agent using Condition, Phase, and RS and their interactions as fixed effects. To isolate effects occurring prior to any pre-programmed low investment, we also analyzed returns in rounds preceding the low investment trials only (rounds 1-11 in the pre-exposure phase and rounds 1-12 in the post-exposure phase) using the same model specification. Finally, to rigorously assess participants' reactions to and recovery from the specific instance of pre-programmed low investment, we conducted an event study analysis centered on the low investment round (\(t=0\)). We analyzed percentage returns in a three-round window (\(t-1\) to \(t+1\)) using a linear mixed-effects model with Phase, Condition, Time Point, and RS Group as fixed effects. We specifically examined two key behavioral responses: the Drop (change in return from \(t-1\) to the low investment round) and the Recovery (change in return from the low investment round to \(t+1\)). The full specification of all statistical models can be found in the supplement.

All models were estimated using the \texttt{afex} package (Singmann et al., 2022) in R. We determined the random effects structure by starting with the maximal model and simplifying until convergence was achieved, ensuring the optimal structure (Matuschek et al., 2017). A similar process was applied to the models analyzing HMM agent investments and participant ratings. We report differences in marginal means rather than effect sizes, as there is no consensus on effect size calculation for mixed models. \(F\)-tests used the Kenward-Roger approximation for degrees of freedom. The Investment variable was Z-transformed to facilitate the interpretation of main effects in the presence of interactions. Significant interactions were probed using planned contrasts with the \texttt{emmeans} package. We applied the ``Sidak'' correction for multiple comparisons to control the familywise error rate while maintaining statistical power.

\section{Behavioral Results}\label{behavioral-results}

\subsection{Analysis of Participant Returns}\label{analysis-of-participant-returns}

On average, investments and returns, as shown in Figure
\ref{fig:gamesPlot}, fell within the documented range of 40-60\% of the
endowment for investments and 35-50\% of the total yield for returns, as
reported in previous studies (Charness et al., 2008; Fiedler et al., 2011).

\begin{figure}

{\centering \includegraphics[width=\textwidth]{article_files/figure-latex/gamesPlot-1} 

}

\caption{Averages and standard errors of the trustee's return as a percentage of the multiplied investment received (y-axis) by Condition, Phase, and game round (x-axis) averaged across RS groups. The blue line shows the returns in the Pre phase and the green line those in the Post phase. The left Panel shows returns in the Control condition and the right one those in the Manipulation condition. The dotted lines identify the rounds where the pre-programmed one-off low investment occurs. We note lower average returns post vs pre in the Manipulation condition, whilst returns in the Control condition are similar between the two phases.}\label{fig:gamesPlot}
\end{figure}

Participants returned higher percentages in the Pre phase compared to
the Post phase
(\(F(1, 201.63) = 5.81\), \(p = .017\)). This
effect was moderated by Condition
(\(F(1, 201.63) = 4.38\), \(p = .038\)):
contrary to our expectations, participants in the Manipulation condition
decreased their returns from pre to post
(\(\Delta M = 0.03\), 95\% CI \([0.01, 0.05]\), \(t(201.50) = 3.15\), \(p = .002\)),
while those in the Control condition showed no change (Figure
\ref{fig:boxPlots}). RS did not moderate this Condition × Phase
interaction.

Higher investments elicited higher percentage returns, indicating
positive reciprocity
(\(F(1, 5955.67) = 325.35\), \(p < .001\)).
This relationship was stronger in the Control condition than in the
Manipulation condition
(\(F(1, 5955.67) = 13.92\), \(p < .001\)).
The effect of investment on returns varied by RS group and Phase
(\(F(1, 5864.62) = 7.84\), \(p = .005\)),
and a four-way interaction indicated that these patterns further
differed across Conditions
(\(F(1, 5864.62) = 9.24\), \(p = .002\)).

\begin{figure}[H]

{\centering \includegraphics[width=\textwidth]{article_files/figure-latex/boxPlots-1} 

}

\caption{Marginal means of percentage returns (top) and HMM investments (bottom) by Phase and Condition. Bars show estimated marginal means; error bars represent 95\% confidence intervals. Participants in the Manipulation condition returned lower proportions post-exposure compared to pre-exposure (** p < .01), while Control participants showed no change. HMM investment did not differ across Phases or Conditions.}\label{fig:boxPlots}
\end{figure}

To examine this four-way interaction, we conducted a contrast analysis
of how the effect of investment on returns changed from pre- to
post-exposure for different RS groups in both conditions (Figure \ref{fig:fourwayInteractionPlot}). Starting
with the Manipulation condition, for participants with low RS, the
effect of investment on returns increased significantly from pre- to
post-phase,
\(\Delta M = 0.03\), \(95\%\ \mathrm{CI}_\mathrm{\scriptstyle Sidak(3)}\) \([0.01, 0.05]\), \(t(5881.28) = 3.15\), \(p_\mathrm{\scriptstyle Sidak(3)} = .005\).
This suggests that after the manipulation, low RS participants became
more responsive to their co-player's investments, returning
proportionally more as investments increased. In contrast, for
participants with high RS, the effect of investment on returns decreased
significantly from pre- to post-exposure,
\(\Delta M = -0.02\), \(95\%\ \mathrm{CI}_\mathrm{\scriptstyle Sidak(3)}\) \([-0.04, 0.00]\), \(t(5891.81) = -2.67\), \(p_\mathrm{\scriptstyle Sidak(3)} = .023\).
This indicates that high RS participants became less responsive to their
co-player's investments after the manipulation, with smaller increases
in returns as investments increased. The difference in these pre-post
changes between high and low RS groups was significant,
\(\Delta M = -0.05\), \(95\%\ \mathrm{CI}_\mathrm{\scriptstyle Sidak(3)}\) \([-0.08, -0.02]\), \(t(5887.33) = -4.11\), \(p_\mathrm{\scriptstyle Sidak(3)} < .001\).
This result suggests that the manipulation had significantly different
effects on how low and high RS participants responded to their
co-player's investments.

In the Control condition, we observed no significant changes in how
participants responded to their co-player's investments between the pre
and post phases, regardless of their RS level.

\begin{figure}[H]
\includegraphics[width=\textwidth]{article_files/figure-latex/fourwayInteractionPlot-1} \caption{Marginal effect of investment on percentage returns by Phase, Condition, and RS group. Bars show estimated slopes (change in returns per SD increase in investment) from the mixed model; error bars represent 95\% confidence intervals. In the Manipulation condition, Low RS participants became more responsive to investments post-exposure (* p < .05), while High RS participants became less responsive (* p < .05). No significant changes were observed in the Control condition.}\label{fig:fourwayInteractionPlot}
\end{figure}

\subsubsection{Returns Prior to Pre-Programmed Low Investment Trials}\label{returns-prior-to-pre-programmed-low-investment-trials}

To distinguish between contrast effects and betrayal aversion as explanations for reduced cooperation in the Manipulation condition, we examined returns in the rounds preceding the pre-programmed low investment. If contrast effects were operating, participants in the Manipulation condition should already show reduced returns before experiencing any low investment in the post-exposure phase. Conversely, if betrayal aversion were the primary mechanism, group differences should only emerge after the low investment.

The Phase \(\times\) Condition interaction was significant in rounds prior to the low investment (\(F(1, 204.05) = 4.86\), \(p = .029\)). Participants in the Manipulation condition significantly decreased their returns from pre- to post-exposure phase even before encountering the low investment (\(\Delta M = 0.03\), 95\% CI \([0.01, 0.06]\), \(t(202.02) = 2.83\), \(p = .005\)), whereas those in the Control condition showed no change (\(\Delta M = 0.00\), 95\% CI \([-0.02, 0.02]\), \(t(201.17) = -0.15\), \(p = .878\)). The four-way interaction also remained significant (\(F(1, 4454.34) = 7.12\), \(p = .008\)), suggesting that the differential responsiveness to investments observed in the full analysis was likewise present before the low investment occurred.

\subsubsection{Reaction to Pre-programmed Low Investment: Event Study Analysis}\label{reaction-to-pre-programmed-low-investment-event-study-analysis}

To understand how participants reacted to and recovered from the pre-programmed low investment, an event study analysis was conducted centered on the low investment round (Figure \ref{fig:eventStudyPlot}). Two behavioral responses were examined: the Drop (change in returns at the low investment round relative to \(t-1\), where negative values indicate reduced returns reflecting punishment of low trust) and Recovery (change in returns at \(t+1\) relative to the low investment round, where positive values indicate restored cooperation).

\begin{figure}[H]
\includegraphics[width=\textwidth]{article_files/figure-latex/eventStudyPlot-1} \caption{Drop and Recovery responses to pre-programmed low investment in the post-exposure phase (pre-exposure phase not shown as no between-condition differences were observed). Drop = change in returns from low investment round minus t-1 (negative values indicate reduced returns); Recovery = change from t+1 minus low investment round (positive values indicate restored returns). Bars show estimated marginal means from the mixed model; error bars represent 95\% confidence intervals. The bracket shows the significant between-condition difference in Recovery for High RS participants. ** p < .01.}\label{fig:eventStudyPlot}
\end{figure}

In the pre-exposure phase, there were no significant differences between conditions in either the Drop (\(M = -0.02\), 95\% CI \([-0.09, 0.05]\), \(t(1773.82) = -0.61\), \(p = .543\)) or Recovery (\(M = 0.00\), 95\% CI \([-0.06, 0.07]\), \(t(1773.56) = 0.11\), \(p = .914\)), confirming that both groups started with equivalent behavioral patterns. In the post-exposure phase, a divergence emerged. The Control group showed a significantly larger Drop than the Manipulation group (\(M = -0.08\), 95\% CI \([-0.15, -0.02]\), \(t(1773.93) = -2.46\), \(p = .014\)), indicating that participants exposed to the forgiving agent showed a blunted immediate reaction to the low investment and did not reduce their returns as sharply. The subsequent Recovery did not differ significantly between conditions overall (\(M = 0.07\), 95\% CI \([0.00, 0.13]\), \(t(1774.02) = 1.92\), \(p = .055\)).

When examining moderation by RS, the pattern appeared to be driven primarily by high RS participants. Low RS participants showed no significant difference in Recovery between conditions (\(M = 0.00\), 95\% CI \([-0.09, 0.10]\), \(t(1774.24) = 0.06\), \(p = .954\)), with both groups displaying modest, similar recovery patterns after the low investment. The larger Drop observed in the Control condition for this group was partly attributable to their elevated cooperation level at \(t-1\).

High RS participants showed a different pattern. In the Control condition, they demonstrated trust repair by significantly increasing their returns after the low investment (\(\Delta M = 0.14\), 95\% CI \([0.08, 0.21]\), \(t(1774.05) = 4.19\), \(p < .001\)). However, those in the Manipulation condition failed to recover, showing no significant increase in returns at \(t+1\) (\(\Delta M = 0.01\), 95\% CI \([-0.05, 0.08]\), \(t(1773.54) = 0.42\), \(p = .676\)). The difference in Recovery between conditions was significant (\(M = 0.13\), 95\% CI \([0.03, 0.22]\), \(t(1773.79) = 2.66\), \(p = .008\)).

In summary, the forgiveness intervention appeared to dampen reciprocal responsiveness, hindering the re-establishment of cooperation following a temporary withdrawal of trust. This effect was more pronounced among high RS individuals. While high RS participants in the Control condition demonstrated active reciprocity by reducing returns sharply when trust was withdrawn and increasing them when trust was restored, those exposed to the forgiving agent exhibited a disengaged pattern characterized by a muted reaction to the low investment and a failure to reinstate high returns afterward.

\subsubsection{HMM Investor in Pre and Post Phases}\label{hmm-investor-in-pre-and-post-phases}

Was the HMM's strategy similar between pre and post phases in the
control condition? Was participants' behavior post exposure
differentiated enough to induce a different reaction from the HMM? To
answer these questions, we test for differences in the HMM agent's
investment by Phase, Condition and RS using a linear mixed-effects model
as described in the methods section. As seen in Figure
\ref{fig:boxPlots}, we find no main or interaction effects, indicating
the HMM's behavior was on aggregate similar across levels of Phase,
Condition and RS. This consistency in the investor's behavior is a
desirable feature of the HMM agent when the participants behavior is
largely simialr between phases. More importantly, it indicates that the
lower returns of participants in the post phase of the manipulation
condition were not differentiated enough to make the HMM react by
transitioning to lower latent trust states. It is also noteworthy that
the HMM agent was relatively successful in imitating human behavior in
this paradigm: When asked during debrief whether they thought the
investors they faced were human or not, \(41\)\% of participants thought
they were either facing a human or were not sure of the nature of the
co-player. When asked to justify their choice, many answers reflected
participants projecting human traits such as ``spitefulness'' or ``greed''
onto the artificial co-player's behavior.

\subsubsection{Exposure Phase Trials}\label{exposure-phase-trials}

So far we focused on analysing behavior for the pre and post phases.
Here, we look at returns and investments in the exposure phase. The
linear mixed effects model of participants' returns in the exposure
phase does not show a main effect of Condition on returns. There was a
main effect of Investment,
\(F(1, 4117.20) = 233.19\), \(p < .001\), with
participants positively reciprocating higher investments, an interaction
effect between Condition and Investment
\(F(1, 4117.20) = 45.93\), \(p < .001\),
showing a stronger positive reciprocity in the Control condition, and
finally a three way interaction between the RS group, Condition and
Investment
\(F(1, 4117.20) = 4.21\), \(p = .040\),
showing that this stronger positive reciprocity to investment in the
Control condition is higher for participants with high RS. The linear
mixed effects model of the HMM investments shows a main effect of
Condition \(F(1, 202) = 197.64\), \(p < .001\),
suggesting higher overall investments for the forgiving HMM compared to
the human-like HMM, but no difference in investments when facing low and
high RS groups.

In summary, despite the forgiving HMM sending overall higher investments
in the exposure phase, participants returned similar proportions of the
multiplied investments as those facing the human-like HMM. The positive
reciprocity of returns to investments was higher in the Control
condition with this relationship stronger for the high RS group.

\subsubsection{Questionnaire Scores and Performance}\label{questionnaire-scores-and-performance}

Whilst we found a significant correlation between participant's Levels
of Personality Functioning Score (LPFS) and the Rejection Sensitivity
Questionnaire score (RSQ), Spearman's \(r_{\mathrm{s}} = .52\), \(p < 0.001\),
there was no correlation between these questionnaire scores and
participant's return or overall task performance.

\subsection{Player Ratings}\label{player-ratings}

Figure \ref{fig:plotRatings} shows participants' ratings of co-players across phases. We examined two contrasts: pre-exposure versus exposure phase ratings, and pre-exposure versus post-exposure ratings.

High RS participants showed more differentiated perceptions of the agents. In the Manipulation condition, they rated the forgiving agents as more cooperative during exposure (\(\Delta M = 2.57\), 95\% CI \([0.84, 4.30]\), \(t(808) = 2.91\), \(p = .004\)). In the Control condition, however, high RS participants rated the same human-like HMM progressively more negatively---lower on cooperation (\(\Delta M = -2.65\), 95\% CI \([-4.37, -0.94]\), \(t(808) = -3.04\), \(p = .002\)), forgiveness (\(\Delta M = -2.19\), 95\% CI \([-4.00, -0.39]\), \(t(808) = -2.38\), \(p = .017\)), and willingness to play again (\(\Delta M = -3.62\), 95\% CI \([-5.73, -1.50]\), \(t(808) = -3.36\), \(p < .001\))---despite the agent's strategy remaining unchanged. Low RS participants showed largely undifferentiated perceptions between pre and exposure phases regardless of condition.

Comparing pre to post-exposure ratings revealed a contrast effect: after experiencing the forgiving agent, both RS groups in the Manipulation condition rated the post-exposure agent (identical to pre-exposure) more negatively on forgiveness (High RS: \(\Delta M\) = -0.88, \(SE\) = 0.38, \(t\)(808.0) = -2.33, p = .020; Low RS: \(\Delta M\) = -1.14, \(SE\) = 0.38, \(t\)(808.0) = -2.98, p = .003) and willingness to play again (High RS: \(\Delta M\) = -1.29, \(SE\) = 0.44, \(t\)(808.0) = -2.92, p = .004; Low RS: \(\Delta M\) = -1.30, \(SE\) = 0.45, \(t\)(808.0) = -2.90, p = .004). Low RS participants in the Control condition showed stable ratings, accurately perceiving the consistent agent strategy, while high RS participants in the Control condition continued their negative drift (Cooperation: \(\Delta M\) = -0.96, \(SE\) = 0.36, \(t\)(808.0) = -2.70, p = .007). These rating patterns converge with the behavioral findings, suggesting high RS individuals are particularly sensitive to relative comparisons between interaction partners.

\begin{figure}[H]

{\centering \includegraphics[width=\textwidth]{article_files/figure-latex/plotRatings-1} 

}

\caption{Participants' ratings of co-players by phase, condition, and RS group. Blue: perceived cooperation; red: perceived forgiveness; green: willingness to play again. Low RS participants showed stable ratings in the Control condition. High RS participants showed declining ratings in the Control condition despite unchanged agent strategy, and more differentiated perceptions in the Manipulation condition.}\label{fig:plotRatings}
\end{figure}

\section{Discussion}\label{discussion}

We used a randomized controlled online experiment where participants
played a RTG with artificial agents designed to simulate human-like
trust-building scenarios. Participants were then exposed to either
forgiving HMM agents (which, by design, were also more cooperative due to their inability to remain in a low-trust state) or standard human-like HMM agents before playing another RTG. We found
that RS did not moderate participants' returns as trustees in the
repeated trust game. While previous research has shown that RS affects
\emph{trust} formation, appraisal and repair, its impact on \emph{reciprocity} in
repeated economic exchanges has been less explored. Our results suggest
a potential dissociation between RS's known effects on broader social
behavior and its limited influence on reciprocity in structured,
repeated interactions, challenging assumptions about the pervasive
influence of RS on social behavior and highlighting the complexity of
factors influencing reciprocity in economic exchanges.

Contrary to our hypothesis, exposure to forgiving agents did not
increase participant's reciprocity or cooperation, nor did it prompt the
artificial agent to increase their trust in participants through higher
investments. Instead, participants reduced their returns overall whilst
the returns of those in the Control group did not change between the pre
and post phase of the experiment. Why did participants reduce their
returns even though they were repeatedly exposed to agents designed to be more forgiving? A look at how the participants rated
their co-players might shed some light on what might be driving this
reduction in returns for those in the Manipulation condition. Those
exposed to the forgiving agent rated their opponent in the post-exposure
phase lower on all attributes even though they faced the same dynamic
human-like HMM as pre-exposure. One possible explanation for this drop
in rating is that participants exhibited a negative contrast effect.
This occurs when the evaluation of a person, object, or situation is
influenced by comparisons with recently encountered contrasting objects
or people. If we've repeatedly interacted with someone exceptionally
nice, our perception of a normal level of niceness might be skewed,
making typical behavior seem less favourable or even negative by
comparison (Kobre \& Lipsitt, 1972). As the most recently faced opponents
were more forgiving (and consequently more cooperative), this negative contrast effect may
have trumped any learning transfer from being repeatedly exposed to
forgiving agents (Zentall, 2005). If this contrast effect
is indeed replicable, then an avenue for future research would be to use
it to our benefit by making the participants play agents with low
cooperation perception.

An alternative explanation relates to differences in exposure-phase experiences: the Control agent performed pre-programmed low investments (round 5 of each game), whereas the Manipulation agent did not. Consequently, Manipulation participants experienced 21 consecutive rounds without any low investments before encountering the post-phase low investment---a substantially longer period of consistently positive interactions than Control participants. This raises the possibility that reduced cooperation in the Manipulation condition reflects heightened betrayal aversion rather than negative contrast effects.

However, the analysis of returns prior to the pre-programmed low investment provides evidence against betrayal aversion as the primary mechanism. If betrayal aversion were driving the effect, group differences should only emerge after participants encountered the low investment in the post-exposure phase. Instead, Manipulation participants had already significantly reduced their returns in rounds 1-12 of the post-exposure phase---before any low investment occurred. This pattern is consistent with contrast effects operating from the beginning of the post-exposure phase, as participants immediately perceived the human-like agent as less cooperative compared to the forgiving agent they had just experienced. While betrayal aversion may contribute to specific aspects of the observed patterns, such as the impaired recovery following the low investment in high RS participants, it cannot account for the overall reduction in cooperation that was already evident before any low investment.

While the negative contrast effect operated across RS groups, the pattern of responses to trust violations differed in ways that align with clinical models of rejection sensitivity. In the event study analysis, high and low RS participants showed comparable immediate reactions to the low investment (the Drop), indicating intact detection of trust violations regardless of RS level. However, the groups diverged in their subsequent recovery patterns: high RS participants in the Manipulation condition failed to restore cooperation following the low investment, whereas low RS participants and Control participants showed recovery. This dissociation between intact rejection detection and impaired relationship repair is consistent with research on social learning difficulties in individuals with elevated RS and related clinical presentations. Studies of borderline personality disorder, where RS is characteristically elevated, have documented specific deficits in updating social expectations following positive interpersonal experiences (Schuster et al., 2021; Staebler et al., 2011). Similarly, research on depression has identified ``cognitive immunization'' processes whereby negative schemas resist modification despite contradictory evidence (Kube et al., 2020). The high RS participants' failure to recover cooperation, despite prior exposure to consistently forgiving behavior, may reflect analogous difficulties in leveraging positive social experiences to update expectations and restore trust.

The four-way interaction findings further support this interpretation. High RS participants showed decreased responsiveness to their co-player's investments following the forgiveness manipulation, a pattern suggestive of withdrawal from contingent social exchange. This reduced sensitivity to partner behavior parallels the self-silencing and social withdrawal documented in high RS populations, where anticipatory self-protection can paradoxically undermine relationship maintenance (Ayduk et al., 2000; Romero-Canyas et al., 2010). In contrast, low RS participants showed increased responsiveness to investments post-phase, suggesting they internalized the cooperative norms experienced during exposure and carried this forward to subsequent interactions. This differential capacity to benefit from positive social experiences maps onto broader findings that RS impedes the acquisition and transfer of adaptive interpersonal strategies (Pietrzak et al., 2005).

The combination of blunted responsiveness to investments and absent recovery in high RS participants suggests a pattern of passive disengagement rather than active retaliation. While overall return levels did not differ between RS groups, these specific behavioral signatures indicate that RS does modulate particular aspects of cooperative behavior. The ratings data complement these findings: high RS participants showed more negative explicit evaluations of their co-players, rating them lower on forgiveness and willingness to interact again. This convergence between explicit ratings and behavioral patterns suggests that the effects of RS on social exchange are expressed across multiple response systems. The structured nature of the trust game may constrain RS effects to specific behavioral signatures (such as contingent responding and recovery) rather than overall cooperation levels, while the more open-ended nature of rating tasks allows for broader expression of RS-related evaluative biases (Lieberman, 2007).

These findings have implications for interventions aimed at promoting trust and cooperation. The present results suggest that exposure to positive social models alone may be insufficient for high RS individuals, and may even produce iatrogenic effects through negative contrast. The specific deficits observed---impaired recovery from trust violations and reduced sensitivity to partner behavior---point to potential intervention targets. Approaches that focus on enhancing the capacity to update expectations following interpersonal ruptures may be more effective than simply providing positive experiences. This could include explicit training in recognizing repair attempts, practicing graduated trust restoration, or developing metacognitive awareness of the tendency toward disengagement following perceived rejection. Future research should examine whether these behavioral patterns generalize to naturalistic social contexts and whether targeted interventions can modify the updating and recovery deficits observed in high RS participants (Balliet et al., 2011).

\subsection{Limitations}\label{limitations}

While this study offers valuable insights into trust and cooperation
dynamics, several limitations warrant consideration. First, and most importantly, the Control and Manipulation conditions
differed not only in the forgiveness of exposure-phase agents but also
in whether participants experienced pre-programmed low investments
during exposure (Control agents sent low investments at round 5 of each exposure game; Manipulation agents did not). This design feature means we cannot fully disentangle the effects of agent forgiveness per se from the effects of experiencing an uninterrupted sequence of positive interactions. Although the analysis of pre-low investment returns provides evidence favoring contrast effects over betrayal aversion as the primary mechanism, and both conditions ultimately faced the same human-like agent in the post-exposure phase, caution is warranted in attributing the observed effects specifically to ``forgiveness exposure.'' Future replications
should equate these experiences across conditions to isolate the effects
of agent forgiveness from the absence of negative experiences. Second, our extreme groups design for RS (selecting
participants with RSQ scores \textgreater{} 15 or \textless{} 10) maximized power to detect
moderation effects but may inflate effect sizes and limits
generalizability to individuals with moderate RS. Future research should
examine RS as a continuous variable. Third, the brief exposure phase
(three 7-round games) may have been insufficient to induce lasting
changes. Fourth, the online format eliminates social cues present in
face-to-face interactions. Notably, while 41\% of participants believed
they faced humans and a similar proportion were unsure, the observed
effects emerged even among those who suspected AI, suggesting robustness
of the findings. Despite these limitations, subsequent studies could
address these constraints by incorporating face-to-face interactions,
longer exposure periods, and continuous RS measurement.

\subsection{Constraints on Generality}\label{constraints-on-generality}

The results may be specific to adults with high or low RS recruited from
online platforms, and may not generalize to clinical populations,
children, or older adults. Our use of a computerized Repeated Trust Game
with HMM agents, while allowing for high experimental control, may limit
generalizability to face-to-face interactions or games with different
economic structures. The brief exposure phase and pre-programmed
low investments are specific to our design and may not reflect real-world
trust-building scenarios. The online context may not capture all aspects
of high-stakes or information-rich interactions. We believe the core
finding of decreased cooperation after exposure to forgiving agents
should generalize across different populations and contexts, though the
effect's strength and its interaction with RS may vary. While the
specific economic game, agent representation, and perception assessment
questions could be varied, the use of artificial agents with consistent
behavior, an exposure phase with more forgiving behavior, and assessment
of both behavior and perceptions should remain constant to preserve the
results. Future studies could systematically vary these factors to
establish the boundaries of generalizability for our findings.
Additionally, cultural differences in norms of cooperation and trust may
influence the generalizability of these findings, necessitating
cross-cultural replications to establish the universality of the
observed effects.

\section{Conclusion}\label{conclusion}

This randomised controlled experiment enabled us to uncover unexpected
effects of exposure to forgiving behavior on subsequent cooperation,
particularly in relation to RS. These findings challenge existing
assumptions about fostering cooperative behavior and suggest the need
for more nuanced interventions. Importantly, the use of HMM-based
artificial agents in this study represents a significant methodological
advancement. By providing a balance between experimental control and
realistic, adaptive behavior, these agents allowed for a nuanced
exploration of trust dynamics that would be challenging to achieve with
human confederates or simplistic computer algorithms. This approach
opens up new possibilities for studying complex social interactions in
controlled settings, potentially bridging the gap between laboratory
experiments and real-world social dynamics.

\pagebreak

\section*{Author contributions statement}\label{author-contributions-statement}
\addcontentsline{toc}{section}{Author contributions statement}

I. Guennouni, G. Koppe and C. Korn. designed and developed the study
concept. Experiment design, testing and data collection were performed
by I. Guennouni. I. Guennouni analysed and interpreted the data under
the supervision of G. Koppe and C. Korn. All authors jointly wrote and
approved the final version of the manuscript for submission.

\section*{Funding}\label{funding}
\addcontentsline{toc}{section}{Funding}

This study was supported by the Federal Ministry of Science, Education,
and Culture (MWK) of the state of Baden-Württemberg within the AI Health
Innovation Cluster Initiative, the German Research Foundation (DFG)
within the Excellence Strategy EXC 2181/1 -- 390900948 (STRUCTURES), and
the Hector II foundation. I. Guennouni was supported by a research
fellowship from the AI Health Innovation Cluster.

\section*{Competing interests statement}\label{competing-interests-statement}
\addcontentsline{toc}{section}{Competing interests statement}

The author(s) declared that there were no conflicts of interest with
respect to the authorship or the publication of this article.

\section*{Acknowledgements}\label{acknowledgements}
\addcontentsline{toc}{section}{Acknowledgements}

We are grateful to Tobias Nolte, Andreas Hula and Read Montague's team
at Virginia Tech for sharing with us the data on the RTG allowing us to
fit the HMM to investor data. We're also grateful for Samuel Dupret for
help designing the online platform used for the experiment and Maarten
Speekenbrink for his guidance on the use of HMM models.

\section*{Additional information}\label{additional-information}
\addcontentsline{toc}{section}{Additional information}

\subsection*{Correspondence}\label{correspondence}
\addcontentsline{toc}{subsection}{Correspondence}

All correspondence and requests for materials should be addressed to I.
Guennouni.

\subsection*{Transparency and data availability}\label{transparency-and-data-availability}
\addcontentsline{toc}{subsection}{Transparency and data availability}

Preregistration: The hypotheses and methods were not preregistered. The primary hypothesis, based on attachment theory, predicted that forgiveness exposure would increase cooperation. The alternative predictions regarding contrast effects and RS-specific updating biases were incorporated into the theoretical framework following initial data analysis, though both perspectives were grounded in existing literature. The analysis plan was not preregistered. Materials: All study materials are
publicly available (\url{https://github.com/ismailg/exposure-public}). Data:
All primary data are publicly available
(\url{https://github.com/ismailg/exposure-public}). Analysis scripts: All
analysis scripts are publicly available
(\url{https://github.com/ismailg/exposure-public}).

\section*{References}\label{references}
\addcontentsline{toc}{section}{References}

\protect\phantomsection\label{refs}
\begin{CSLReferences}{1}{0}
\bibitem[\citeproctext]{ref-abramov_influence_2022-1}
Abramov, G., Kautz, J., Miellet, S., \& Deane, F. P. (2022). The {Influence} of {Attachment Style}, {Self-protective Beliefs}, and {Feelings} of {Rejection} on the {Decline} and {Growth} of {Trust} as a {Function} of {Borderline Personality Disorder Trait Count}. \emph{Journal of Psychopathology and Behavioral Assessment}, \emph{44}(3), 773--786. \url{https://doi.org/10.1007/s10862-022-09965-9}

\bibitem[\citeproctext]{ref-almaatouq_empirica_2021}
Almaatouq, A., Becker, J., Houghton, J. P., Paton, N., Watts, D. J., \& Whiting, M. E. (2021). Empirica: A virtual lab for high-throughput macro-level experiments. \emph{Behavior Research Methods}, \emph{53}(5), 2158--2171. \url{https://doi.org/10.3758/s13428-020-01535-9}

\bibitem[\citeproctext]{ref-american_psychiatric_association_diagnostic_2013}
American Psychiatric Association. (2013). \emph{Diagnostic and {Statistical Manual} of {Mental Disorders}} (Fifth Edition). American Psychiatric Association. \url{https://doi.org/10.1176/appi.books.9780890425596}

\bibitem[\citeproctext]{ref-ayduk_rejection_2000}
Ayduk, O., Downey, G., \& Kim, M. (2000). Rejection sensitivity and depressive symptoms in women. \emph{Personality and Social Psychology Bulletin}, \emph{26}(8), 909--919. \url{https://doi.org/10.1177/0146167200269001}

\bibitem[\citeproctext]{ref-balliet_reward_2011}
Balliet, D., Mulder, L. B., \& Van Lange, P. A. M. (2011). Reward, punishment, and cooperation: {A} meta-analysis. \emph{Psychological Bulletin}, \emph{137}(4), 594--615. \url{https://doi.org/10.1037/a0023489}

\bibitem[\citeproctext]{ref-bandura_social_1977}
Bandura, A. (1977). \emph{Social {Learning Theory}}. Prentice Hall.

\bibitem[\citeproctext]{ref-berenson_rejection_2009}
Berenson, K. R., Gyurak, A., Ayduk, Ö., Downey, G., Garner, M. J., Mogg, K., Bradley, B. P., \& Pine, D. S. (2009). Rejection sensitivity and disruption of attention by social threat cues. \emph{Journal of Research in Personality}, \emph{43}(6), 1064--1072. \url{https://doi.org/10.1016/j.jrp.2009.07.007}

\bibitem[\citeproctext]{ref-bowlby_attachment_1978}
Bowlby, J. (1978). Attachment theory and its therapeutic implications. \emph{Adolescent Psychiatry}, \emph{6}, 5--33.

\bibitem[\citeproctext]{ref-bowlby_secure_1988}
Bowlby, J. (1988). \emph{A secure base: Parent-child attachment and healthy human development}. Basic Books.

\bibitem[\citeproctext]{ref-charness_investment_2008}
Charness, G., Cobo-Reyes, R., \& Jiménez, N. (2008). An investment game with third-party intervention. \emph{Journal of Economic Behavior \& Organization}, \emph{68}(1), 18--28. \url{https://doi.org/10.1016/j.jebo.2008.02.006}

\bibitem[\citeproctext]{ref-downey_implications_1996}
Downey, G., \& Feldman, S. I. (1996). Implications of rejection sensitivity for intimate relationships. \emph{Journal of Personality and Social Psychology}, \emph{70}(6), 1327--1343. \url{https://doi.org/10.1037/0022-3514.70.6.1327}

\bibitem[\citeproctext]{ref-downey_early_1997}
Downey, G., Khouri, H., \& Feldman, S. I. (1997). Early interpersonal trauma and later adjustment: {The} mediational role of rejection sensitivity. In \emph{Developmental perspectives on trauma: {Theory}, research, and intervention} (pp. 85--114). University of Rochester Press.

\bibitem[\citeproctext]{ref-faul_statistical_2009}
Faul, F., Erdfelder, E., Buchner, A., \& Lang, A.-G. (2009). Statistical power analyses using {G}*{Power} 3.1: {Tests} for correlation and regression analyses. \emph{Behavior Research Methods}, \emph{41}(4), 1149--1160. \url{https://doi.org/10.3758/BRM.41.4.1149}

\bibitem[\citeproctext]{ref-fiedler_social_2011}
Fiedler, M., Haruvy, E., \& Li, S. X. (2011). Social distance in a virtual world experiment. \emph{Games and Economic Behavior}, \emph{72}(2), 400--426. \url{https://doi.org/10.1016/j.geb.2010.09.004}

\bibitem[\citeproctext]{ref-fonagy_mentalizing_2017}
Fonagy, P., \& Campbell, C. (2017). \href{https://www.ncbi.nlm.nih.gov/pubmed/29135441}{Mentalizing, attachment and epistemic trust: How psychotherapy can promote resilience}. \emph{Psychiatria Hungarica: A Magyar Pszichiatriai Tarsasag Tudomanyos Folyoirata}, \emph{32}(3), 283--287.

\bibitem[\citeproctext]{ref-fowler_cooperative_2010}
Fowler, J. H., \& Christakis, N. A. (2010). Cooperative behavior cascades in human social networks. \emph{Proceedings of the National Academy of Sciences}, \emph{107}(12), 5334--5338. \url{https://doi.org/10.1073/pnas.0913149107}

\bibitem[\citeproctext]{ref-gao_associations_2017}
Gao, S., Assink, M., Cipriani, A., \& Lin, K. (2017). Associations between rejection sensitivity and mental health outcomes: {A} meta-analytic review. \emph{Clinical Psychology Review}, \emph{57}, 59--74. \url{https://doi.org/10.1016/j.cpr.2017.08.007}

\bibitem[\citeproctext]{ref-hepp_prosociality_2022}
Hepp, J., \& Niedtfeld, I. (2022). Prosociality in personality disorders: {Status} quo and research agenda. \emph{Current Opinion in Psychology}, \emph{44}, 208--214. \url{https://doi.org/10.1016/j.copsyc.2021.09.013}

\bibitem[\citeproctext]{ref-herpertz_social-cognitive_2014}
Herpertz, S. C., \& Bertsch, K. (2014). The social-cognitive basis of personality disorders. \emph{Current Opinion in Psychiatry}, \emph{27}(1), 73--77. \url{https://doi.org/10.1097/YCO.0000000000000026}

\bibitem[\citeproctext]{ref-joyce_trust_1995}
Joyce, B., Dickhaut, J., \& McCabe, K. (1995). Trust, {Reciprocity}, and {Social History}. \emph{Games and Economic Behavior}, \emph{10}(1), 122--142.

\bibitem[\citeproctext]{ref-king-casas_getting_2005}
King-Casas, B., Tomlin, D., Anen, C., Camerer, C. F., Quartz, S. R., \& Montague, P. R. (2005). Getting to {Know You}: {Reputation} and {Trust} in a {Two-Person Economic Exchange}. \emph{Science}, \emph{308}(5718), 78--83. \url{https://doi.org/10.1126/science.1108062}

\bibitem[\citeproctext]{ref-kobre_negative_1972}
Kobre, K. R., \& Lipsitt, L. P. (1972). A negative contrast effect in newborns. \emph{Journal of Experimental Child Psychology}, \emph{14}(1), 81--91. \url{https://doi.org/10.1016/0022-0965(72)90033-1}

\bibitem[\citeproctext]{ref-kube_distorted_2020}
Kube, T., Schwarting, R., Rozenkrantz, L., Glombiewski, J. A., \& Rief, W. (2020). Distorted cognitive processes in major depression: A predictive processing perspective. \emph{Biological Psychiatry}, \emph{87}(5), 388--398. \url{https://doi.org/10.1016/j.biopsych.2019.07.017}

\bibitem[\citeproctext]{ref-lieberman_social_2007}
Lieberman, M. D. (2007). Social {Cognitive Neuroscience}: {A Review} of {Core Processes}. \emph{Annual Review of Psychology}, \emph{58}(1), 259--289. \url{https://doi.org/10.1146/annurev.psych.58.110405.085654}

\bibitem[\citeproctext]{ref-matuschek_balancing_2017}
Matuschek, H., Kliegl, R., Vasishth, S., Baayen, H., \& Bates, D. (2017). Balancing {Type I} error and power in linear mixed models. \emph{Journal of Memory and Language}, \emph{94}, 305--315. \url{https://doi.org/10.1016/j.jml.2017.01.001}

\bibitem[\citeproctext]{ref-miano_rejection_2013}
Miano, A., Fertuck, E. A., Arntz, A., \& Stanley, B. (2013). Rejection {Sensitivity Is} a {Mediator Between Borderline Personality Disorder Features} and {Facial Trust Appraisal}. \emph{Journal of Personality Disorders}, \emph{27}(4), 442--456. \url{https://doi.org/10.1521/pedi_2013_27_096}

\bibitem[\citeproctext]{ref-mikulincer_attachment_1998}
Mikulincer, M. (1998). Attachment working models and the sense of trust: {An} exploration of interaction goals and affect regulation. \emph{Journal of Personality and Social Psychology}, \emph{74}(5), 1209--1224. \url{https://doi.org/10.1037/0022-3514.74.5.1209}

\bibitem[\citeproctext]{ref-mulder_relationship_1999}
Mulder, R. T., Joyce, P. R., Sullivan, P. F., Bulik, C. M., \& Carter, F. A. (1999). The relationship among three models of personality psychopathology: {DSM-III-R} personality disorder, {TCI} scores and {DSQ} defences. \emph{Psychological Medicine}, \emph{29}(4), 943--951. \url{https://doi.org/10.1017/S0033291799008533}

\bibitem[\citeproctext]{ref-pietrzak_appearance-rejection_2005}
Pietrzak, R. H., Downey, G., \& Ayduk, O. (2005). Appearance-rejection sensitivity predicts body dysmorphic disorder symptoms and cosmetic surgery. \emph{Annals of Clinical Psychiatry}, \emph{17}(4), 213--219. \url{https://doi.org/10.1080/10401230500295471}

\bibitem[\citeproctext]{ref-richetin_emotional_2018}
Richetin, J., Poggi, A., Ricciardelli, P., Fertuck, E. A., \& Preti, E. (2018). The emotional components of rejection sensitivity as a mediator between {Borderline Personality Disorder} and biased appraisal of trust in faces. \emph{Clinical Neuropsychiatry: Journal of Treatment Evaluation}, \emph{15}(4), 200--205.

\bibitem[\citeproctext]{ref-romero-canyas_rejection_2010}
Romero-Canyas, R., Downey, G., Berenson, K. R., Ayduk, O., \& Kang, N. J. (2010). Rejection sensitivity and the rejection-hostility link in romantic relationships. \emph{Journal of Personality}, \emph{78}(1), 119--148. \url{https://doi.org/10.1111/j.1467-6494.2009.00611.x}

\bibitem[\citeproctext]{ref-schuster_ambiguous_2021}
Schuster, F., Hoerz-Sagstetter, S., Seidl, E., Mauer, C., Zeiss, M., Reinhard, M. A., Padberg, F., \& Jobst, A. (2021). Ambiguous social rejection from a close other affects neural and behavioral responses in borderline personality disorder. \emph{Personality Disorders: Theory, Research, and Treatment}, \emph{12}(6), 583--594. \url{https://doi.org/10.1037/per0000454}

\bibitem[\citeproctext]{ref-singmann_afex_2022}
Singmann, H., Bolker, B., Westfall, J., Aust, F., Ben-Shachar, M. S., Højsgaard, S., Fox, J., Lawrence, M. A., Mertens, U., Love, J., Lenth, R., \& Christensen, R. H. B. (2022). \emph{Afex: {Analysis} of {Factorial Experiments}}.

\bibitem[\citeproctext]{ref-staebler_facial_2011}
Staebler, K., Renneberg, B., Stopsack, M., Fiedler, P., Weiler, M., \& Roepke, S. (2011). Facial emotional expression in reaction to social exclusion in borderline personality disorder. \emph{Psychological Medicine}, \emph{41}(9), 1929--1938. \url{https://doi.org/10.1017/S0033291711000080}

\bibitem[\citeproctext]{ref-weekers_level_2019}
Weekers, L. C., Hutsebaut, J., \& Kamphuis, J. H. (2019). The {Level} of {Personality Functioning Scale}-{Brief Form} 2.0: {Update} of a brief instrument for assessing level of personality functioning. \emph{Personality and Mental Health}, \emph{13}(1), 3--14. \url{https://doi.org/10.1002/pmh.1434}

\bibitem[\citeproctext]{ref-zentall_within-trial_2005}
Zentall, T. R. (2005). A {Within-trial Contrast Effect} and its {Implications} for {Several Social Psychological Phenomena}. \emph{International Journal of Comparative Psychology}, \emph{18}(4). \url{https://doi.org/10.46867/ijcp.2005.18.04.08}

\end{CSLReferences}

\end{document}
